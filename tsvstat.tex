% !TEX TS-program = pdflatex
% !TEX encoding = UTF-8 Unicode

% This is a simple template for a LaTeX document using the "article" class.
% See "book", "report", "letter" for other types of document.

\documentclass[11pt]{article} % use larger type; default would be 10pt

\usepackage[utf8]{inputenc} % set input encoding (not needed with XeLaTeX)

%%% Examples of Article customizations
% These packages are optional, depending whether you want the features they provide.
% See the LaTeX Companion or other references for full information.

%%% PAGE DIMENSIONS
\usepackage{geometry} % to change the page dimensions
\geometry{a4paper} % or letterpaper (US) or a5paper or....
% \geometry{margin=2in} % for example, change the margins to 2 inches all round
% \geometry{landscape} % set up the page for landscape
%   read geometry.pdf for detailed page layout information

\usepackage{graphicx} % support the \includegraphics command and options

% \usepackage[parfill]{parskip} % Activate to begin paragraphs with an empty line rather than an indent

%%% PACKAGES
\usepackage{booktabs} % for much better looking tables
\usepackage{array} % for better arrays (eg matrices) in maths
\usepackage{paralist} % very flexible & customisable lists (eg. enumerate/itemize, etc.)
\usepackage{verbatim} % adds environment for commenting out blocks of text & for better verbatim
\usepackage{subfig} % make it possible to include more than one captioned figure/table in a single float
% These packages are all incorporated in the memoir class to one degree or another...

%%% HEADERS & FOOTERS
\usepackage{fancyhdr} % This should be set AFTER setting up the page geometry
\pagestyle{fancy} % options: empty , plain , fancy
\renewcommand{\headrulewidth}{0pt} % customise the layout...
\lhead{}\chead{}\rhead{}
\lfoot{}\cfoot{\thepage}\rfoot{}

%%% SECTION TITLE APPEARANCE
\usepackage{sectsty}
\allsectionsfont{\sffamily\mdseries\upshape} % (See the fntguide.pdf for font help)
% (This matches ConTeXt defaults)

\usepackage{amsmath}

%%% ToC (table of contents) APPEARANCE
\usepackage[nottoc,notlof,notlot]{tocbibind} % Put the bibliography in the ToC
\usepackage[titles,subfigure]{tocloft} % Alter the style of the Table of Contents
\renewcommand{\cftsecfont}{\rmfamily\mdseries\upshape}
\renewcommand{\cftsecpagefont}{\rmfamily\mdseries\upshape} % No bold!

%%% END Article customizations

%%% The "real" document content comes below...

\title{Tsvstat}
\author{wozgonon}
\date{} % Activate to display a given date or no date (if empty),
         % otherwise the current date is printed 

\begin{document}
\maketitle

\section{Introduction}

The {\it tsvtstat} tool can be used to generate a table of summary statistics given a table or stream of tab separated numeric observations.


\section{Statistics}

The formulas used to generate statistics are compatible with those used by common spreadsheets.

\subsection{Mean}

\begin{align*}
\mu&=\sum \frac{x_i}{n}  \\
\implies  \sum x_i &= n\mu & \text{(1)}
\end{align*}

\subsection{Sample Variance}

\begin{align*}
s^2&= \frac{1}{n-1}\sum_1^n (x_i-\mu)^2 \\
              &=\frac{1}{n-1}\sum_1^n (x_i^2 - 2\mu x_i +\mu^2) \\
              &=\frac{1}{n-1}(\sum_1^n x_i^2 - 2\mu \sum x_i +n\mu^2) \\
              &=\frac{1}{n-1}(\sum_1^n x_i^2 - 2n\mu^2 +n\mu^2) & \text{by (1)}\\
              &=\frac{1}{n-1}\sum_1^n x_i^2 - n\mu^2) \\
\end{align*}


\subsection{Skewness}
\begin{align*}
skew &= \frac{n}{(n-1)(n-2)}\sum\frac{(x,-\mu)^3}{s^3} \\
        &= \frac{n}{(n-1)(n-2)}\frac{\sum x_i^3 - 3\mu\sum x_i^2 + 2n\mu^3}{s^3} & \text{subst. (2)}\\
\\
\sum_1^n {(x_i - \mu)^3} &= \sum_1^n (x_i^2 - 2x_i\mu + \mu^2)(x_i - \mu) \\
                                &= \sum_1^n (x_i^3 - 2x_i^2\mu + x_i\mu^2 - \mu x_i^2 + 2x_i\mu^2 - \mu^3) \\
                                &= \sum_1^n x_i^3 - 2\mu\sum x_i^2 + \mu^2\sum x_i - \mu\sum x_i^2 + 2\mu^2\sum x_i - n\mu^3 \\
                                &= \sum_1^n x_i^3 - 3\mu\sum x_i^2 + 3\mu^2\sum x_i - n\mu^3 \\
                                &= \sum_1^n x_i^3 - 3\mu\sum x_i^2 + 3n\mu^3 - n\mu^3 & \text{by (1)} \\
                                &= \sum_1^n x_i^3 - 3\mu\sum x_i^2 + 2n\mu^3 & \text{(2)} \\
\end{align*}

\subsection{Kurtosis}
\begin{align*}
kurt &= \frac{n(n+1)}{(n-1)(n-2)(n-3)}\sum_1^n\frac{(x,-\mu)^4}{s^4} - \frac{3(n-1)^2}{(n-2)(n-3)}\\
      &= \frac{n(n+1)}{(n-1)(n-2)(n-3)}\sum_1^n\frac{(\sum_1^n x_i^4 - 4\mu\sum x_i^3 + 3\mu^2\sum x_i^2 + 2\mu^4(1-n) }{s^4} & \text{subst. (3)}\\
      & - \frac{3(n-1)^2}{(n-2)(n-3)} \\
\\
\sum_1^n {(x_i - \mu)^4} &= (x_i - \mu)^3(x_i - \mu) \\
                                &= \sum_1^n x_i^3 - 3\mu\sum x_i^2 + 2n\mu^3)(x_i - \mu) & \text{subst. (2)}\\
                                &= \sum_1^n x_i^4 - 3\mu\sum x_i^3 + 2n\mu^3\sum x_i  - \mu \sum_1^n x_i^3 + 3\mu^2\sum x_i^2 - 2n\mu^4 \\
                                &= \sum_1^n x_i^4 - 4\mu\sum x_i^3 + 2\mu^4 + 3\mu^2\sum x_i^2 - 2n\mu^4 \\
                                &= \sum_1^n x_i^4 - 4\mu\sum x_i^3 + 3\mu^2\sum x_i^2 + 2\mu^4(1-n) & \text{(3)}
\end{align*}

\end{document}
